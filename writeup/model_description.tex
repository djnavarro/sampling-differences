% Options for packages loaded elsewhere
\PassOptionsToPackage{unicode}{hyperref}
\PassOptionsToPackage{hyphens}{url}
%
\documentclass[
  english,
  doc]{apa6}
\usepackage{lmodern}
\usepackage{amssymb,amsmath}
\usepackage{ifxetex,ifluatex}
\ifnum 0\ifxetex 1\fi\ifluatex 1\fi=0 % if pdftex
  \usepackage[T1]{fontenc}
  \usepackage[utf8]{inputenc}
  \usepackage{textcomp} % provide euro and other symbols
\else % if luatex or xetex
  \usepackage{unicode-math}
  \defaultfontfeatures{Scale=MatchLowercase}
  \defaultfontfeatures[\rmfamily]{Ligatures=TeX,Scale=1}
\fi
% Use upquote if available, for straight quotes in verbatim environments
\IfFileExists{upquote.sty}{\usepackage{upquote}}{}
\IfFileExists{microtype.sty}{% use microtype if available
  \usepackage[]{microtype}
  \UseMicrotypeSet[protrusion]{basicmath} % disable protrusion for tt fonts
}{}
\makeatletter
\@ifundefined{KOMAClassName}{% if non-KOMA class
  \IfFileExists{parskip.sty}{%
    \usepackage{parskip}
  }{% else
    \setlength{\parindent}{0pt}
    \setlength{\parskip}{6pt plus 2pt minus 1pt}}
}{% if KOMA class
  \KOMAoptions{parskip=half}}
\makeatother
\usepackage{xcolor}
\IfFileExists{xurl.sty}{\usepackage{xurl}}{} % add URL line breaks if available
\IfFileExists{bookmark.sty}{\usepackage{bookmark}}{\usepackage{hyperref}}
\hypersetup{
  pdftitle={Appendix B: Model description},
  pdflang={en-EN},
  hidelinks,
  pdfcreator={LaTeX via pandoc}}
\urlstyle{same} % disable monospaced font for URLs
\usepackage{graphicx,grffile}
\makeatletter
\def\maxwidth{\ifdim\Gin@nat@width>\linewidth\linewidth\else\Gin@nat@width\fi}
\def\maxheight{\ifdim\Gin@nat@height>\textheight\textheight\else\Gin@nat@height\fi}
\makeatother
% Scale images if necessary, so that they will not overflow the page
% margins by default, and it is still possible to overwrite the defaults
% using explicit options in \includegraphics[width, height, ...]{}
\setkeys{Gin}{width=\maxwidth,height=\maxheight,keepaspectratio}
% Set default figure placement to htbp
\makeatletter
\def\fps@figure{htbp}
\makeatother
\setlength{\emergencystretch}{3em} % prevent overfull lines
\providecommand{\tightlist}{%
  \setlength{\itemsep}{0pt}\setlength{\parskip}{0pt}}
\setcounter{secnumdepth}{-\maxdimen} % remove section numbering
% Make \paragraph and \subparagraph free-standing
\ifx\paragraph\undefined\else
  \let\oldparagraph\paragraph
  \renewcommand{\paragraph}[1]{\oldparagraph{#1}\mbox{}}
\fi
\ifx\subparagraph\undefined\else
  \let\oldsubparagraph\subparagraph
  \renewcommand{\subparagraph}[1]{\oldsubparagraph{#1}\mbox{}}
\fi
% Manuscript styling
\usepackage{upgreek}
\captionsetup{font=singlespacing,justification=justified}

% Table formatting
\usepackage{longtable}
\usepackage{lscape}
% \usepackage[counterclockwise]{rotating}   % Landscape page setup for large tables
\usepackage{multirow}		% Table styling
\usepackage{tabularx}		% Control Column width
\usepackage[flushleft]{threeparttable}	% Allows for three part tables with a specified notes section
\usepackage{threeparttablex}            % Lets threeparttable work with longtable

% Create new environments so endfloat can handle them
% \newenvironment{ltable}
%   {\begin{landscape}\begin{center}\begin{threeparttable}}
%   {\end{threeparttable}\end{center}\end{landscape}}
\newenvironment{lltable}{\begin{landscape}\begin{center}\begin{ThreePartTable}}{\end{ThreePartTable}\end{center}\end{landscape}}

% Enables adjusting longtable caption width to table width
% Solution found at http://golatex.de/longtable-mit-caption-so-breit-wie-die-tabelle-t15767.html
\makeatletter
\newcommand\LastLTentrywidth{1em}
\newlength\longtablewidth
\setlength{\longtablewidth}{1in}
\newcommand{\getlongtablewidth}{\begingroup \ifcsname LT@\roman{LT@tables}\endcsname \global\longtablewidth=0pt \renewcommand{\LT@entry}[2]{\global\advance\longtablewidth by ##2\relax\gdef\LastLTentrywidth{##2}}\@nameuse{LT@\roman{LT@tables}} \fi \endgroup}

% \setlength{\parindent}{0.5in}
% \setlength{\parskip}{0pt plus 0pt minus 0pt}

% \usepackage{etoolbox}
\makeatletter
\patchcmd{\HyOrg@maketitle}
  {\section{\normalfont\normalsize\abstractname}}
  {\section*{\normalfont\normalsize\abstractname}}
  {}{\typeout{Failed to patch abstract.}}
\makeatother
\shorttitle{Model description}
\author{Danielle Navarro\textsuperscript{1, 2}}
\affiliation{
\vspace{0.5cm}
\textsuperscript{1} School of Psychology, UNSW Sydney\\\textsuperscript{2} UNSW Data Science Hub, UNSW Sydney}
\authornote{

Correspondence concerning this article should be addressed to Danielle Navarro, School of Psychology, UNSW Sydney, Kensington NSW 2052. E-mail: d.navarro@unsw.edu.au}
\usepackage{lineno}

\linenumbers
\usepackage{csquotes}
\usepackage{bm}
\ifxetex
  % Load polyglossia as late as possible: uses bidi with RTL langages (e.g. Hebrew, Arabic)
  \usepackage{polyglossia}
  \setmainlanguage[]{english}
\else
  \usepackage[shorthands=off,main=english]{babel}
\fi

\title{Appendix B: Model description}

\date{}

\begin{document}
\maketitle

The model used in this paper is a simplified version of the one described by Hayes et al (2019, see their Appendix B), and whose implementation is archived at \url{https://github.com/djnavarro/samplingframes}. The motivation for the model is to consider how one might learn an arbitrary function \(f\) defined over some stimulus space \(\mathcal{X}\). In the general case \(\mathcal{X}\) can be richly structured, but in the current setting we can consider stimuli that vary
only on a single continuous dimension. The Gaussian process (GP; see Rasmussen \& Williams 2006) provides a method
for specifying priors over smooth functions \(f : \mathbb{R} \rightarrow \mathbb{R}\) that maps every possible stimulus \(x\) onto a subjective inductive strength \(y = f(x)\). The function \(f\) is defined over the entire stimulus space \(\mathcal{X}\) but is measured only at a known, finite set of points \(\bm{x} = (x_1, \ldots, x_n)\). In this setting the joint distribution over the corresponding outputs \(\bm{y} = (y_1, \ldots, y_n)\) is a finite-dimensional marginalisation of the Gaussian process and is thus a multivariate Gaussian with mean vector \(\bm{\mu} = (\mu_1, \ldots, \mu_n)\) and covariance matrix \(\bm{\Sigma} = [\sigma_{ij}]\) for \(i, j \in 1, \ldots, n\)

\[
f(x) \sim \mbox{Normal}(\bm\mu, \bm\Sigma)
\]

The covariance matrix is used to control the smoothness of the inferred function \(f\) and defined by the kernel function \(\sigma_{ij} = K(x_i, x_j)\). Following the logic outlined by Hayes et al, we apply a radial basis kernel function kernel in which the correlation diminishes as a Gaussian function of the distance \(d_{ij}\) between items in psychological space:

\[
K(x_i, x_j) = \tau^2 \exp\left(-\rho {d_ij}^2\right)
\]
and
\[
\sigma_{ij} = \left\{ \begin{array}{rl} K(x_i, x_j) & \mbox{ if } i \neq j \\ K(x_i, x_i) + \sigma^2 & \mbox{ if } i = j \end{array}  \right.
\]
In these expressions, \(\tau\) describes the baseline correlation between stimuli, \(\rho\) governs the rate at which the correlation decays as a function of psychological distance, and \(\sigma\) is the inherent noise in the data. In our applications we assume the prior mean is the same for all stimuli, so the fourth and last free parameter of the model is the prior mean \(\mu\).

\begin{itemize}
\tightlist
\item
  Hayes, B. K., Banner, S., Forrester, S. and Navarro, D. J. (2019). Selective sampling and inductive inference: Drawing inferences based on observed and missing evidence. \emph{Cognitive Psychology, 113}. \url{https://doi.org/10.1016/j.cogpsych.2019.05.003}
\item
  Rasmussen, C. E., and Williams, C. K. I. (2006). \emph{Gaussian Processes for Machine Learning}
\end{itemize}

\end{document}
